\newgeometry{top=2cm,left=2cm,right=2cm,bottom=2cm} 

\thispagestyle{empty}

\selectlanguage{english} 
%
\begin{abstract}
\noindent
Since Einstein proposed his theory of general relativity in 1915 physicists have searched for black hole solutions of all kinds. In 1992, Máximo Bañados, Claudio Teitelboim, and Jorge Zanelli discovered one such solution in a (2+1) dimensional spacetime with negative curvature called the BTZ Black Hole. In this thesis we construct the BTZ Black Hole from AdS$_3$ and analyse its properties. We derie the metric and find that the rotating BTZ black hole possesses an inner and an outer event horizon at $r^2_{\pm}
=  \frac{\ell^2}{2} \, \left[
M \pm \sqrt{M^2 - \left(\frac{J}{\ell} \right)^2}
\right]$, where $M$ and $J$ are the mass and angular momentum of the black hole respectively, and $\ell$ is related to the cosmological constant by $\Lambda = - \ell^{-2}$.
This is very similar to the (3+1) dimensional Kerr Black Hole. We also find that apart from the behaviour at infinity, the causal structure of the static BTZ Black Hole resembles that of the Schwarzschild black hole quite closely.\\

\noindent
Lastly, following the approach of \cite{Green} closely, we utilise that the static BTZ Black Hole can be constructed by a identification of points in 3 dimensional anti-de Sitter space, allowing for the exact computation of Green's functions on the static BTZ spacetime. Using the KMS condition, we show that the properties of these Green's functions are exactly those of a thermal Green's function, implying that the static BTZ Black Hole has a non-zero temperature of $T =  \frac{r_+}{2 \pi \ell^2}$. 
\end{abstract}
%
%
%
\selectlanguage{danish} 
%
\begin{abstract}
\noindent
Siden Einstein fremførte sin teori om generel relativitetsteori i 1915 har fysikere søgt efter løsninger der opfører sig som sorte huller. I 1992 opdagede Máximo Bañados, Claudio Teitelboim, and Jorge Zanelli en sådan løsning i en (2+1) dimensionel rumtid med negativ krumning kaldet et BTZ sort hul. I dette bachelorprojekt konstrueres det BTZ sorte hul ud fra AdS$_3$ og det egenskaber analyseres. Vi udleder metrikken og ser at det roterende BTZ sorte hul besidder både en indre og en ydre begivenhedshorisont ved $r^2_{\pm}
=  \frac{\ell^2}{2} \, \left[
M \pm \sqrt{M^2 - \left(\frac{J}{\ell} \right)^2}
\right]$, hvor $M$ og $J$ er henholdsvis massen og det angulære moment af det sorte hul, og $\ell$ er relateret til den kosmologiske konstant ved $\Lambda = - \ell^{-2}$.
Dette er meget lig det (3+1) dimensionale Kerr sorte hul. Vi finder desuden, at på nær opførelsen ved uendelig, at den kausale struktur af det statiske BTZ sorte hul har mange ligheder med den kausale struktur af Schwarzschild sorte huller.\\

\noindent
Til slut følger vi udledningen givet i \cite{Green} tæt og udnytter, at det statiske BTZ sorte hul kan konstrueres ved identifikation of punkter i 3 dimensionalt anti-de Sitter space, hvilket muliggør en eksakt udregning af Green's funktioner på den statiske BTZ rumtid. Ved brug af KMS betingelsen vises det, at disse Green's funktioner har præcis de egenskaber vi forventer for en termisk Green's funktion, hvilket medfører at det statiske BTZ sorte hul har temperaturen $T =  \frac{r_+}{2 \pi \ell^2}$.

\end{abstract}
%
\selectlanguage{english}