% \newgeometry{top=4cm,left=4cm,right=4cm,bottom=4cm} 
%
\section{Conclusion and outlook}
As explained in the introductory part of the thesis, the ultimate aim of our work was twofold. Firstly, we wanted to find explicit expressions for the two-point functions between the scalar chiral primary operators $\mathcal{O}_{Z} = \tr Z^L$, $\mathcal{O}_{\bar{Z}} = \tr \bar{Z}^L$ and $\mathcal{O}_{X} = \tr X^L$, in the dCFT setups arising from D3-D7 probe-brane configurations. Both the disconnected tree-level and the disconnected 1-loop contributions to these two-point functions, in both the $SO(3) \times SO(3)$ and the $SO(5)$ symmetric setups, were easily obtainable by direct extension of the works \cite{One-point functions in D3-D7, One-point functions in D3-D7 SO(5)}. For the connected tree-level constribution however, in contrast to the $SO(3)$ symmetric D5-D3 probe brane setup studied in \cite{Two-point functions in D5-D3}, the infinite sums obtained in the $SO(3) \times SO(3)$ symmetric setup are seemingly unevaluable. Thus, we could not in general find a closed form expression for the connected tree-level contributions to the chiral primary two-point functions. We were however able to evaluate the infinite sums for very short chiral primary two-point functions; specifically for $L_1  = 2$, $L_2 = 2$. We were also able to reproduce and correct\footnote{With our analysis, we were able to reproduce the connected tree-level contributions find in \cite{Two-point functions in D5-D3}, only for $L_1$, $L_2$ even. For $L_1$, $L_2$ odd, we very clearly obtain different, although similar results. We belive this to be an oversight on the part of the authors.} the results found in \cite{Two-point functions in D5-D3}, for the cases of $k_1 = 1$ and $k_2 = 1$ units of external gauge-field flux through the $S^2 \times S^2$. In conclusion, we were able to find explicit expressions to first order in $\lambda$, for the chiral primary two-point functions in the $SO(3) \times SO(3)$ symmetric setup, to the extend of seeming possibility. For the $SO(5)$ symmetric setup, only the disconnected tree-level and the disconnected 1-loop contributions to the chiral primary two-point functions have been found so far.\\
Our second ultimate aim was to extend the work of \cite{Length L length 2 two-point functions D5-D3} from the D5-D3 probe-brane setup to the D3-D7 probe-brane setups. In other words, we wanted to also find explicit expressions for two-point functions between short scalar operators of the form $\mathcal{O}_{W_1 W_2} = \tr[W_1 W_2]$, and Bethe state operators $\mathcal{O}_{L} = \Psi_M^{i_1 \ldots i_L} \tr[V_{i_1} \cdots V_{i_L}]$, in the $SO(3) \times SO(3)$ and $SO(5)$ symmetric dCFT setups. The insight presented in \cite{Length L length 2 two-point functions D5-D3} was that the connected tree-level contribution to these two-point functions could be re-expressed as an overlap of the form $\bra{\text{MPS}} Q_{W_1 W_2} \ket{\Psi_M}$. This overlap can then be further reduced down to the overlap $\braket{\text{MPS}}{\Psi_M}$, which in the $SO(3)$ symmetric D5-D3 setup can be expressed in a compact determinant form \cite{non-protected one-point functions}. In the case of the $SO(3) \times SO(3)$ symmetric setup, we find that it is also possible to re-express the connected tree-level contribution to these long / short two-point functions in terms of spin-chain operators $Q_{W_1 W_2}$, but only when $W_2 = \bar{W}_1$. If we want to insist on a spin-chain picture for $W_2 \neq \bar{W}_1$, we can think of the associated $Q_{W_1 W_2}$ as mapping the $\mathfrak{su}(2)$ spin-chain states out of the $\mathfrak{su}(2)$ subsector. For the $SO(3) \times SO(3)$ symmetric setup, we furthermore have the problem that the third conserved charge $Q_3$ of the $\mathfrak{su}(2)$ subsector, does not annihilate the MPS. The consequence of this is first of all that only a subset of the $\bra{\text{MPS}} Q_{W_1 W_2} \ket{\Psi_M}$ overlaps with $W_2 = \bar{W}_1$ can be written in terms of the simpler $\braket{\text{MPS}}{\Psi_M}$ overlap. Even more problematic, is the fact that $\braket{\text{MPS}}{\Psi_M}$ is not even known for general values of $k_1$, $k_2$ and $M$. For certain specific parameter values, $\braket{\text{MPS}}{\Psi_M}$ has however been computed in the $SO(3) \times SO(3)$ symmetric setup, and the results can be found in \cite{Lack of integrability in SO(3)xSO(3)}. In further conclusion, we were indeed able to find explicit expressions for the long / short two-point functions in the $SO(3) \times SO(3)$ symmetric setup, but only for very specific choices of $W_1$, $W_2$, due in part to the more complicated nature of the MPS. The computations and results for long / short two-point functions in the $SO(5)$ symmetric setup remains to be investigated.\\
With the contributions to first order in $\lambda$ to the these different two-point functions now at hand (\textit{at least for the case of $SO(3) \times SO(3)$} symmetric vevs), we have made some significant progress towards a very non-trivial check of the AdS / CFT duality. Looking forward, it would be very interesting to first precisely identify and subsequently compute the objects on the gravity side, dual to these two-point functions. One might somewhat intuitively expect these dual objects to somehow be related to string configurations, with two string-ends connected to the $AdS_5$ boundary, and one string-end connected to the D7 brane. To the best of our knowledge however, there currently exists no concrete attempts to compute the area of such string configurations, nor any other suggestions to what the gravity dual objects might be. Needless to say, there is certainly more work to be done pertaining to an AdS / CFT check through the various dCFT two-point functions presented in this thesis.


%
%
%
\label{end of main body}