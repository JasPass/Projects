% \newgeometry{top=4cm,left=4cm,right=4cm,bottom=4cm} 
%
\thispagestyle{empty}
%
%
\section*{Abstract}
We study various two-point functions in certain defect versions of $\mathcal{N} = 4$ super Yang Mills theory. These defect theories are obtained by insertion of a D7 probe-brane, with either $AdS_4 \times S^2 \times S^2$ or $AdS_4 \times S^4$ geometry, into the standard D3 brane configuration of AdS / CFT. The $\mathcal{N} = 4$ SYM theories, arising from the decoupling limit of these brane configurations, have non-zero vacuum expectation values (\textit{vevs}) for the scalar fields $\phi_i$. These non-zero vevs breaks super symmetry completely and conformal symmetry partially, thus presenting us with an interesting opportunity to make non-trivial tests of the AdS / CFT duality.\\
We focus first on two-point functions with $SO(3) \times SO(3)$ symmetric vevs, between chiral primary operators of the forms $\tr Z^L$, $\tr \bar{Z}^L$, $\tr X^L$, where $X = \phi_1 + i \phi_4$, $Y = \phi_2 + i \phi_5$ and $Z = \phi_3 + i \phi_6$. By use of pertubative methods, we were able to reduce the connected tree-level contributions to these two-point functions, down to expressions involving complicated infinite sums. These infinite sums unfortunately seem unevaluable in general. However, for specific values of $L$ and the parameters associated to stabilization of the brane configurations, we were able to explicitly evaluate the infinite sums.\\
We also study two-point functions, first with $SO(3) \times SO(3)$ symmetric vevs, between short scalar operators $\mathcal{O}_{W_1 W_2} = \tr[W_1 W_2]$ with scalars $W_1,W_2 = X,Y,Z, \bar{X}, \bar{Y}, \bar{Z}$, and Bethe state operators $\mathcal{O}_{L} = \Psi_M^{i_1 \ldots i_L} \tr[V_{i_1} \cdots V_{i_l}]$, with $V_i = X,Z$ and $\Psi_M$ being a Bethe wavefunction with $M$ excitations. By use of integrability techniques, we find that certain choices of $W_1,W_2$ allows for the tree-level contribution to these two-point functions to be expressed in terms of the tree-level value of $\expval{\mathcal{O}_{L}}$. The computations of these various types of two-point functions provide the first step towards a very non-trival check of the AdS / CFT duality. We hope that future work will enable us to complete this endeavor, by studying the corresponding objects on the gravity side of the duality. 

%The types of two-point functions subject to analysis are 1. two-point functions between scalar single-trace chiral primary operators, and 2. two-point functions between short single-trace scalar operators and Bethe state operators. By use of pertubative methods, we were able to reduce the tree-level contributions to the chiral primary two-point functions, down to infinite sums which generally seem uncomputable. For very specific parameter values associated to the operators and defect setups, we were able to evaluate the sums, and obtain explicit values for the tree-level contributions to these two-point functions.

\newpage
\thispagestyle{empty}
%
%
\section*{Acknowledgements}
I would like to thank my supervisor \textit{Charlotte Kristjansen}, first for presenting me with the opportunity to write this thesis, and subsequently for taking time to answer my seemingly endless stream of questions regarding the subject of this thesis, as well as AdS/CFT in general. I would also like to thank \textit{Matthias Wilhelm} and \textit{Matthias Volk} for engaging in many helpful discussions, as well as sharing thier valuable knowledge on the topic of this thesis. A big thank you also goes out to \textit{Marius de Leeuw}, who generously shared with me some very helpful Mathematica notebooks, concerning the computation of certain results from \cite{Two-point functions in D5-D3} relevant to this thesis. Similarly, I would also like to thank \textit{Isak Buhl-Mortensen} for his very helpful notes on the reduction of $10D$ Majorana-Weyl fermions to $4D$ Majorana fermions.\\
I would also like to thank my fellow master students for engaging in many stimulating discussions about our respective theses and physics in general. Particular gratitude goes out to \textit{Khalil Idiab}, with whom I have shared \textit{Charlotte Kristjansen} as supervisor. As the contents of our respective works were, and still is, closely connected, his input and suggestions have been of great value to me. Particular gratitude also goes out to \textit{Christian Schi\o tt}, who has been a dear friend to me throughout my time at university. Your help and advice has been invaluable.\\
In addition, I would like to thank my family; my mother in particular, the care and support of whom I am eternally grateful for. Lastly, I would like to thank my dear friend \textit{Peter Gross}. Without your help, this thesis would likely never have been brought to completion. Thank you. 
%
%