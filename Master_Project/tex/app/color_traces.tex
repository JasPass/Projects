% \newgeometry{top=2cm,left=2cm,right=2cm,bottom=2cm} 
%
\section{Trace relations for Lie algebra generators}\label{app:trace_relations}
In writing down the general expressions for the connected tree-level constributions to the chiral primary two-point functions of section \ref{sec:CPO}, the following results from \cite{Two-point functions in D5-D3} was of instrumental importance.
%
%
\begin{equation}
\tr[ t_3^L \hat{Y}^m_{\ell} ] = \alpha^L_\ell \, \delta^{m,0}
\end{equation}
%
%
\begin{equation}
\tr[ t_1^L \hat{Y}^m_{\ell} ]
=
\frac{i^{\ell+m}}{2^\ell}
\sqrt{\frac{\Gamma(\ell+m+1)}{\Gamma(\ell-m+1)}}
\frac{\Gamma(\ell-m+1)}
{
\Gamma \left( \frac{\ell-m+2}{2} \right)
\Gamma \left( \frac{\ell+m+2}{2} \right)
} \,
\alpha^L_\ell
\end{equation}
%
%
In this appendix, we will give a brief review of procedure used in \cite{Two-point functions in D5-D3}, to derive the above formulae. The strategy will be to first write the generator $t_3$ in terms of the fuzzy harmonics $\hat{Y}^m_{\ell}$, and subsequently use the fussion algebra for the $\mathfrak{so}(3)$ fuzzy harmonics to write the product $t_3^L$ as a sum $\mathfrak{so}(3)$ fuzzy harmonics.
%
%
\begin{equation}
t_3^L = \sum_{\ell=0}^L \alpha^L_\ell \, \hat{Y}^0_\ell
\end{equation}
%
%
We then make use of a clever recursion relation between the $\alpha^L_\ell$ coefficients to compute all of them from $\alpha^L_L$. We finally relate the $t_1^L$ trace to the $t_3^L$ trace, using an appropriate similarity transformation.


\subsection[$\mathfrak{so}(3)$ trace relations]{$\mathbf{\mathfrak{so}(3)}$ trace relations}
In order to compute the two-point functions, we need to know the following trace:
%
%
\begin{equation}
\tr[ t_3^L \hat{Y}_\ell^m ]
\end{equation}
%
%
A defining feature of the $\hat{Y}_\ell^m$ matrices is that they obey:
%
%
\begin{equation}
L_3 \hat{Y}_\ell^m = [t_3, \hat{Y}_\ell^m] = m \hat{Y}_\ell^m
\end{equation}
%
%
Let us now attempt to compute the following trace:
%
%
\begin{equation}
\tr[ t_3^L L_3 \hat{Y}_\ell^m ]
=
\tr[ t_3^L [t_3, \hat{Y}_\ell^m] ]
=
\tr[ [t_3^L, t_3] \hat{Y}_\ell^m ]
=
0
\end{equation}
%
%
But we also know that:
%
%
\begin{equation}
\tr[ t_3^L L_3 \hat{Y}_\ell^m ]
=
m \tr[ t_3^L \hat{Y}_\ell^m ]
\end{equation}
%
%
Thus, we conclude that:
%
%
\begin{equation}
\tr[ t_3^L \hat{Y}_\ell^m ] = 0
%
\quad \text{for} \quad
%
m \neq 0
%
\quad \Rightarrow \quad
%
t_3^L = \sum_{\ell=0}^{\infty} \alpha_{\ell}^L \, \hat{Y}_\ell^0
\end{equation}
%
%
Another defining feature of the $\hat{Y}_\ell^m$ matrices is that they obey:
%
%
\begin{equation}
L^2 \hat{Y}_l^m = [t_i, [t^i, \hat{Y}_\ell^m]] = \ell (\ell + 1) \hat{Y}_\ell^m
\end{equation}
%
%
It will be useful to split up $L^2$ in the following standard way:
%
%
\begin{equation}
L^2 = L_- L_+ + L_3^2 + L_3
\end{equation}
%
%
Let us now attempt to compute the following trace:
%
%
\begin{equation}
\tr[ t_3^L L^2 \hat{Y}_\ell^m ]
\end{equation}
%
%
We do this in steps using the split up:
%
%
\begin{equation}
\tr[ t_3^L L_3^2 \hat{Y}_\ell^m ]
=
\tr[ t_3^L [t_3, [t_3, \hat{Y}_\ell^m]] ]
=
\tr[ [t_3^L, t_3] [t_3, \hat{Y}_\ell^m] ]
=
0
\end{equation}
%
%
Now, the non trivial contribution:
%
%
\begin{equation}
\tr[ t_3^L L_- L_+ \hat{Y}_\ell^m ]
=
\tr[ t_3^L [t_-, [t_+, \hat{Y}_\ell^m]] ]
=
\tr[ [t_3^L, t_-] [t_+, \hat{Y}_\ell^m] ]
=
\tr[ [[t_3^L, t_-], t_+] \hat{Y}_\ell^m ]
\end{equation}
%
%

%
%
\begin{equation}
\hat{Y}^{m}_{\ell}
=
\sum_{m_1=-\ell_1}^{\ell_1}
\sum_{m_2=-\ell_2}^{\ell_2}
C^{\ell,m}_{\ell_1,m_1;\ell_2,m_2}
\hat{Y}^{m_1}_{\ell_1} \hat{Y}^{m_2}_{\ell_2}
\end{equation}
%
%

%
%
\begin{equation}
\hat{Y}^{m_1}_{\ell_1} \hat{Y}^{m_2}_{\ell_2}
=
\sum_{\ell=0}^{k-1}
\sum_{m=-\ell}^{\ell}
F^{\ell,m}_{\ell_1,m_1;\ell_2,m_2}
\hat{Y}^{m}_{\ell}
\end{equation}
%
%

%
%
\begin{equation}
m_1 + m_2 = m
%
\quad , \quad
%
|\ell_2 - \ell_1| \leq \ell \leq \ell_1 + \ell_2
\end{equation}
%
%

%
%
\begin{equation}
\left( \begin{array}{c}
\hat{Y}^{0}_{2} \\
\hat{Y}^{0}_{1} \\
\hat{Y}^{0}_{0} \\
\end{array}\right)
=
\left( \begin{array}{ccc}
C^{2,0}_{1,1;1,-1} & C^{2,0}_{1,0;1,0} & C^{2,0}_{1,-1;1,1} \\
C^{1,0}_{1,1;1,-1} & C^{1,0}_{1,0;1,0} & C^{1,0}_{1,-1;1,1} \\
C^{0,0}_{1,1;1,-1} & C^{0,0}_{1,0;1,0} & C^{0,0}_{1,-1;1,1} \\
\end{array}\right)
\left( \begin{array}{c}
\hat{Y}^{1}_{1} \hat{Y}^{-1}_{1} \\
\hat{Y}^{0}_{1} \hat{Y}^{0}_{1} \\
\hat{Y}^{-1}_{1} \hat{Y}^{1}_{1} \\
\end{array}\right)
\end{equation}
%
%

%
%
\begin{equation}
\hat{Y}^{0}_{1} \hat{Y}^{0}_{1}
=
F^{2,0}_{1,0;1,0} \hat{Y}^{0}_{2}
+
F^{1,0}_{1,0;1,0} \hat{Y}^{0}_{1}
+
F^{0,0}_{1,0;1,0} \hat{Y}^{0}_{0}
\end{equation}
%
%
Work in progress...

\subsection[$\mathfrak{so}(5)$ trace relations]{$\mathbf{\mathfrak{so}(5)}$ trace relations}
Work in progress...

